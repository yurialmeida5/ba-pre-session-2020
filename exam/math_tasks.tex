\documentclass[10pt]{article}
 
\usepackage[margin=1in]{geometry} 
\usepackage{amsmath,amsthm,amssymb, graphicx, multicol, array, enumerate, gensymb}
\newcommand{\N}{\mathbb{N}}
\newcommand{\Z}{\mathbb{Z}}
 
\newenvironment{problem}[2][Problem]{\begin{trivlist}
\item[\hskip \labelsep {\bfseries #1}\hskip \labelsep {\bfseries #2.}]}{\end{trivlist}}

\begin{document}
 
\title{Mathematics problems}
\date{}
\maketitle

 \section{Elementary algebra}
 
\begin{problem}{1.1}
Simplify $$\frac{y^{58}}{y^4 \cdot y^{12}}$$
\end{problem}

\begin{problem}{1.2}
Solve for $x$:
$$8^2 \cdot 2^x = 2^9$$
\end{problem}

\begin{problem}{1.3}
Calculate the missing value. If $\frac{x}{y}$ is 3, then $x^{-2}y^{2}=\dots$
\end{problem}

\begin{problem}{1.4}
Calculate
$$\frac{\sqrt{2^{13}}}{\sqrt{8^3}}$$
\end{problem}

\begin{problem}{1.5}
True or False ($x$ and $y$ and $z$ are real numbers):
\begin{enumerate}[(a)]
    \item $x+y=y+x$
    \item $x(y+z)=xy+xz$
    \item $x^{y+z}=x^yx^z$
    \item $\frac{x^y}{x^z}=x^{y-z}$
\end{enumerate}
\end{problem}

\begin{problem}{1.6}
Find the solution for the equality below:
$$\frac{x^2-25}{x-5}=3$$
\end{problem}

\section{Functions of one variable}

\begin{problem}{2.1 (Based on SYD 2.5.6)}
The relationship between temperatures measured in Kelvin and Fahrenheit is linear. 0 K is equivalent to -460\degree F and 1000 K is the same as 1340\degree F.
 Which temperature is measured by the same number on both scales?
\end{problem}

\begin{problem}{2.2}
Take the following function $f(x)=2x+3$. Find y if $f(y)=17$.
\end{problem}

\begin{problem}{2.3}
Find all values of x that satisfy:
$$3^{2x^2-4x+3}=27$$
\end{problem}

\begin{problem}{2.4}
Solve the following problem. If the annual GDP growth of a country is 1\%, how long does it take the economy to double its GDP?
\end{problem}

\begin{problem}{2.5}
Calculate the following value
$$\ln\left(\frac{e^2}{e^3} \right)$$
\end{problem}

\section{Calculus}

\begin{problem}{3.1}
Calculate the following sum
$$\sum\limits_{i=0}^{\infty} \left( \frac{1}{6^i}+0.25^i\right)$$
\end{problem}

\begin{problem}{3.2}
Find the following limit
$$\lim\limits_{x \rightarrow 3}\frac{x^2-9}{x-3}$$
\end{problem}

\begin{problem}{3.3}
Find the slope of the function $f(x)=x^3-4$ at $(-1,-5)$.
\end{problem}

\begin{problem}{3.4}
Find the following derivative
$$\left( \frac{x^2+3}{x+2}\right)'$$
\end{problem}

\begin{problem}{3.5}
Find the second derivative of
 $$f(x)= x^7+4x^2$$
\end{problem}

\begin{problem}{3.6}
Find the derivative of
$$f(x)=\frac{x^4+4^x}{\ln(x)}$$
\end{problem}

\begin{problem}{3.7}
Consider the following function. Find all of its stationary points and classify them as local minima, local maxima or inflection points.
$$f(x)=3x^3-9x$$
\end{problem}

\begin{problem}{3.8}
Let $f(x,y)=x^2+2y^3$. Calculate $f(2,3)$
\end{problem}

\begin{problem}{3.9}
Consider the following function: $f(x,y)=\ln(2x-y)$. For what combinations of $x$ and $y$ is this function defined?
\end{problem}

\begin{problem}{3.10}
Find all partial derivatives of the following function:
$$f(x,y)= x^5e^y+x^2y^3$$
\end{problem}

\begin{problem}{3.11}
Find the local maxima or minima of the following function:
$$f(x,y)=\sqrt{xy}-0.7x-0.7y$$
\end{problem}

\begin{problem}{3.12}
Solve the following constrained optimization problem using Lagrange's method:
$\max x^2y^2$ s.t. $x+y=10$
\end{problem}

\section{Linear algebra}

\begin{problem}{4.1}
Take the following matrices:
$$A=\begin{bmatrix} 2 & 3\\ 4 & 1 \\ 1 & 2\end{bmatrix}$$
$$B=\begin{bmatrix} 1 & 4 & 1\\2 & 1 & 2\end{bmatrix}$$
What is $A \cdot B$?
\end{problem}

\begin{problem}{4.2}
Take the following matrices:
$$A=\begin{bmatrix} 2 & 3\\ 4 & 1 \\ 1 & 2\end{bmatrix}$$
$$B=\begin{bmatrix} 1 & 4 & 1\\2 & 1 & 2\end{bmatrix}$$
What is $B \cdot A$?
\end{problem}

\begin{problem}{4.3}
What is the transpose of the following matrix?
$$\begin{bmatrix}3.3 & 5.1 \\ 6.1 & 1.23 \\ 45.76 & 0\end{bmatrix}$$
\end{problem}

\begin{problem}{4.4}
Calculate the determinant of
$$\begin{bmatrix}2 & 3 & 0\\ 4 & 5 & 2 \\ 2 & 5 & 3\end{bmatrix} $$
\end{problem}

\section{Probability theory}

\begin{problem}{5.1}
You run an experiment where you flip a coin twice. Each time you get either heads (H) or tails (T). What is the sample space of your experiment?
\end{problem}

\begin{problem}{5.2}
You are observing a race with 30 competitors. How many possible outcomes exist for the 1st, 2nd and 3rd place?
\end{problem}

\begin{problem}{5.3}
You run an experiment in which you toss a dice twice and record the results. What is the probability that at least one of the tosses end up being odd?
\end{problem}
\end{document}
